% SPDX-FileCopyrightText: Copyright (c) 2023-2025 Yegor Bugayenko
% SPDX-License-Identifier: MIT

\documentclass{article}
\usepackage{../lecture-notes/notes}
\usepackage{href-ul}
\newcommand*\thetitle{Code Coverage}
\begin{document}

\lnTitlePage{15}{24}{pJrXQ5rptig}

\lnThought{A line of code is covered if it is executed during the test suite’s run.}

\pptBanner{Example, Part I}
\begin{multicols}{2}
Live Code:\par
{\small\begin{ffcode}
(*@\textcolor{green}{int fibonacci(int n) \{} @*)
  (*@\textcolor{green}{if (n <= 0) \{} @*)
    return 0;
  (*@\textcolor{green}{\}} @*)
  (*@\textcolor{green}{if (n <= 2) \{} @*)
    (*@\textcolor{green}{return 1;} @*)
  (*@\textcolor{green}{\}} @*)
  return fibonacci(n-1)
    + fibonacci(n-2);
(*@\textcolor{green}{\}}@*)
\end{ffcode}
}
\par\columnbreak\par
Test Code:\par
{\small\begin{ffcode}
assert fibonacci(1) == 1;
assert fibonacci(2) == 1;
\end{ffcode}
}
\( C = 7/10 = 70\% \)
\end{multicols}
\plush{}

\pptBanner{Example, Part II}
\begin{multicols}{2}
Live Code:\par
{\small\begin{ffcode}
(*@\textcolor{green}{int fibonacci(int n) \{} @*)
  (*@\textcolor{green}{if (n <= 0) \{} @*)
    return 0;
  (*@\textcolor{green}{\}} @*)
  (*@\textcolor{green}{if (n <= 2) \{} @*)
    (*@\textcolor{green}{return 1;} @*)
  (*@\textcolor{green}{\}} @*)
  (*@\textcolor{green}{return fibonacci(n-1)}@*)
    (*@\textcolor{green}{+ fibonacci(n-2);}@*)
(*@\textcolor{green}{\}}@*)
\end{ffcode}
}
\par\columnbreak\par
Test Code:\par
{\small\begin{ffcode}
assert fibonacci(1) == 1;
assert fibonacci(2) == 1;

assert fibonacci(9) == 34;
assert fibonacci(10) == 55;
\end{ffcode}
}
\( C = 9/10 = 90\% \)
\end{multicols}
\plush{}

\lnThought{Not only lines of code may be covered by tests.}

\lnPitch{
  \pptBanner{Some Kinds of Code Coverage}
  \begin{itemize}\setlength\itemsep{0em}
  \item Line Coverage
  \item Statement Coverage
  \item Branch Coverage
  \item Condition Coverage
  \item Function Coverage
  \item Linear Code Sequence and Jump (LCSAJ) Coverage
  \item Modified Condition / Decision Coverage (MC/DC)
  \end{itemize}}

\pptBanner{Four Kinds of Coverage}
\begin{multicols}{2}
Live Code:\par
{\small\begin{ffcode}
int foo(int x) {
  if (x < 0) { return x; }
  if (x > 10 || x == 0) {
    return 42 / x;
  } else {
    return 1;
  }
}
\end{ffcode}
}
\par\columnbreak\par
Test Code:\par
{\small\begin{ffcode}
assert foo(1) == 1;
assert foo(50) == 42;
\end{ffcode}
}
\( C_{\texttt{line}} = 6/6 = 100\% \)\par
\( C_{\texttt{statement}} = 5/6 = 83\% \)\par
\( C_{\texttt{branch}} = 3/4 = 75\% \)\par
\( C_{\texttt{condition}} = 3/5 = 60\% \)\par
\end{multicols}
\plush{}

\lnThought{The coverage level can be an indicator of quality.}

\lnQuote
  [William Robert Elmendorf]
  {william-elmendorf}
  {A disciplined test control process is composed of five steps:
  \begin{inparaenum}[1)]
    \item establish the intended extent of testing;
    \item create a list of functional \ul{variations} eligible for testing;
    \item rank and subset the \ul{eligible} variations so that test resources can be directed at those with the higher payoff;
    \item calculate the \ul{test coverage} of the test case library; and
    \item verify attainment of the \ul{planned} test coverage.
  \end{inparaenum}}
  {elmendorf1969}

\lnQuote
  [David Gelperin]
  {david-gelperin}
  {However, only half regularly document their test designs, only half regularly save their tests for reuse after software changes, and an extremely small five percent provide regular \ul{measurements} of code coverage.}
  {gelperin1988}
\lnPitch{
  \begin{multicols}{2}
  \includegraphics[width=.6\linewidth]{practices.png}
  \par\columnbreak\par
  ``We note an \ul{inconsistency}. A high percentage of the respondents
  felt that the testing in their organization was a systematic
  and organized activity (91\% answered either ``yes''
  or ``sometimes'' to this practice). However, [...]
  an extremely small 5\% provide \ul{regular
  measurements of code coverage}.''\par
  \lnSource{gelperin1988}
  \end{multicols}}

\lnQuote
  [Boris Beizer]
  {boris-beizer}
  {Junky software takes more tests to achieve coverage, but it breaks under any systematic test.}
  {beizer1995}

\lnThought{Not all practitioners believe in high coverage numbers.}

\lnQuote
  [Brian Marick]
  {brian-marick}
  {Coverage numbers (like many numbers) are dangerous because they're \ul{objective} but \ul{incomplete}. They too often distort sensible action. Using them in isolation is as foolish as hiring based only on GPA.}
  {marick1997}

\lnQuote
  [Martin Fowler]
  {martin-fowler}
  {I would be suspicious of anything like 100\% --- it would smell of someone writing tests to make the coverage \ul{numbers happy}, but not thinking about what they are doing.}
  {fowler1997}

\lnQuote
  [Cem Kaner]
  {cem-kaner}
  {As you get near 100 percent line coverage, that doesn't tell you the product is near release. It just tells you that the product is no longer \ul{obviously far} from release according to \ul{this measure}.}
  {kaner2002lessons}

\lnQuote
  [Pavneet Singh Kochhar, David Lo, Julia Lawall, Nachiappan Nagappan]
  {pavneet-singh-kochhar}
  {Our results show that coverage has an \ul{insignificant correlation} with the number of bugs that are found after the release of the software at the project level, and \ul{no such correlation} at the file level.}
  {kochhar2017}

\lnQuote
  [Goran Petrovi{\'c}]
  {goran-petrovic}
  {Google \ul{does not enforce} any code coverage thresholds across the entire codebase. Projects (or groups of projects) are free to define their own thresholds and goals. Many projects opt-into a centralized voluntary alerting system that defines \ul{five levels} of code coverage thresholds.}
  {ivankovi2019}

\lnPitch{
  \pptBanner{Code Coverage Threshold Levels in Google}
  \includegraphics[width=.7\textwidth]{levels.png}}

\lnQuote
  [Adam Bender]
  {adam-bender}
  {Code coverage \ul{does not guarantee} that the covered lines or branches have been tested correctly, it just guarantees that they have been executed by a test. But a low code coverage number \ul{does guarantee} that large areas of the product are going completely untested by automation on every single deployment.}
  {arguelles2020}

\lnThought{Some international standards require coverage control during software development.}

\lnQuote
  {../15-code-coverage/survey}
  {Many \ul{contracts} are specifying that a certain \ul{percentage} of the statements or instructions must be successfully executed before the acceptance of the software by the customer.}
  {bowen1979survey}

\lnPitch{
  \pptBanner{Industry Standards that Require Code Coverage}
  \begin{itemize}
  \item ISO-26262: ``Road Vehicles'' functional safety (Switzerland)
  \item IEC 61508: ``Functional Safety of Electrical/Electronic/Programmable Electronic Safety-related Systems'' (UK)
  \item DO-178C: ``Software Considerations in Airborne Systems and Equipment Certification'' (USA)
  \item IEC 62304: ``Medical Device Software'' (UK)
  \end{itemize}}

\lnPitch{
  \begin{pptWide}{2}
  ISO-26262:\\
  \includegraphics[width=.9\linewidth]{iso-26262.png}\par
  IEC 61508:\\
  \includegraphics[width=.9\linewidth]{iec-61508.png}\par
  \par\columnbreak\par
  DO-178C:\\
  \includegraphics[width=.9\linewidth]{do-178c.png}\par
  IEC 62304:\\
  \includegraphics[width=.9\linewidth]{iso-26262.png}\par
  \end{pptWide}}

\lnThought{Open source tools exist that help keep coverage under control.}

\lnPitch{\pptBanner{Codecov.io}\includegraphics[width=.7\textwidth]{codecov.png}}

\lnPitch{\pptBanner{Line Coverage}\includegraphics[width=.7\textwidth]{lines.png}}

\lnPitch{\pptBanner{Tarpaulin for Rust}\includegraphics[width=.7\textwidth]{tarpaulin.png}}

\pptBanner{Code Coverage Threshold, JaCoCo Example}
\begin{multicols}{2}
{\tiny\begin{ffcode}
<project>
  [...]
  <build>
    <plugins>
      <plugin>
        <groupId>org.jacoco</groupId>
        <artifactId>jacoco-maven-plugin</artifactId>
        <version>0.8.11</version>
        <executions>
          <execution>
            <id>jacoco-initialize</id>
            <goals>
              <goal>prepare-agent</goal>
            </goals>
          </execution>
          <execution>
            <id>jacoco-check</id>
            <goals>
              <goal>check</goal>
            </goals>
            <configuration>
              <rules>
                (*@\textcolor{orange}{[...] \(\leftarrow\) Next slide}@*)
              </rules>
            </configuration>
          </execution>
          <execution>
            <id>report</id>
            <goals>
              <goal>report</goal>
            </goals>
          </execution>
        </executions>
      </plugin>
    </plugins>
  </build>
</project>
\end{ffcode}
}
\end{multicols}
\plush{}

\pptBanner{Code Coverage Threshold, JaCoCo Rules}
\begin{multicols}{2}
{\tiny\begin{ffcode}
<rules>
  <rule>
    <element>BUNDLE</element>
    <limits>
      <limit>
        <counter>(*@\textcolor{orange}{INSTRUCTION}@*)</counter>
        <value>COVEREDRATIO</value>
        <minimum>0.67</minimum>
      </limit>
      <limit>
        <counter>(*@\textcolor{orange}{LINE}@*)</counter>
        <value>COVEREDRATIO</value>
        <minimum>0.84</minimum>
      </limit>
      <limit>
        <counter>(*@\textcolor{orange}{BRANCH}@*)</counter>
        <value>COVEREDRATIO</value>
        <minimum>0.47</minimum>
      </limit>
      <limit>
        <counter>(*@\textcolor{orange}{COMPLEXITY}@*)</counter>
        <value>COVEREDRATIO</value>
        <minimum>0.57</minimum>
      </limit>
      <limit>
        <counter>(*@\textcolor{orange}{METHOD}@*)</counter>
        <value>COVEREDRATIO</value>
        <minimum>0.76</minimum>
      </limit>
      <limit>
        <counter>(*@\textcolor{orange}{CLASS}@*)</counter>
        <value>MISSEDCOUNT</value>
        <maximum>2</maximum>
      </limit>
    </limits>
  </rule>
</rules>
\end{ffcode}
}
\end{multicols}
{\scriptsize Source: \url{https://github.com/volodya-lombrozo/jtcop}\par}
\plush{}

\lnPitch{
  Code Coverage can be calculated by a few tools:
  \begin{itemize}
  \item \href{https://www.jacoco.org}{JaCoCo} for Java
  \item \href{https://istanbul.js.org/}{Istanbul} for Javascript
  \item \href{https://gcc.gnu.org/onlinedocs/gcc/Gcov.html}{Gcov} for C/C++
  \item \href{https://pypi.org/project/coverage/}{Coverage.py} for Python
  \item \href{https://github.com/simplecov-ruby/simplecov}{Simplecov} for Ruby
  \item \href{https://github.com/xd009642/tarpaulin}{Tarpaulin} for Rust
  \end{itemize}}

\end{document}
